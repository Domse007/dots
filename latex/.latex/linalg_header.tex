%---------------------------------------------------------------------
%     Configurations for the header
%---------------------------------------------------------------------
\newcommand{\lecturers}{{\"O}zlem Imamoglu\\ Olga Sorkine-Hornung}
\newcommand{\semester}{2022}
\newcommand{\department}{D-INFK}
\newcommand{\lecturetitle}{Linear Algebra}
\newcommand{\lectureurl}{\url{https://igl.ethz.ch/teaching/linear-algebra/la2022/}}

% header for the assignment
\newcommand{\headerStudentInfo}[3]{
\noindent
\rlap{\department}\hfill \lecturetitle \hfill\llap{HS \semester} \\
\lecturers \par\vskip+\parskip\vspace{12pt} 
\centerline{{\LARGE \assignment }}\vspace{6pt}
\centerline{Name: \textbf{#2}}
\centerline{Legi: \textbf{#3}}
\vspace*{5ex}
}

\headerStudentInfo{1}{Jakob Klemm}{22-921-241}

%---------------------------------------------------------------------
%   setup for the page margins    
%---------------------------------------------------------------------
\usepackage{geometry}
\geometry{
 a4paper,
%  total={170mm,237mm},
 left=20mm,
 right=20mm,
 top=20mm,
 bottom=20mm,
 }
 \setlength{\parindent}{0pt}
 
%---------------------------------------------------------------------
%   packages we need
%---------------------------------------------------------------------
\usepackage[utf8]{inputenc}
\usepackage{times} % change the font typeface
\usepackage{url}

\usepackage{amsmath} % for pmatrix
% \usepackage{enumitem} % for itemize
\usepackage{subfiles}
\usepackage{amscd,amssymb, amsthm,isomath}
\usepackage{mathtools}
\usepackage{dsfont}
\usepackage{comment}
\usepackage{units}
\usepackage{fontawesome}
\usepackage{subfig}
\usepackage{graphicx}
\usepackage{verbatim}
\usepackage{nicefrac}
\usepackage{ifthen}
\usepackage{booktabs}
\usepackage{hyperref}

%---------------------------------------------------------------------
%   Math Environment
%---------------------------------------------------------------------

\providecommand{\C}{} \renewcommand{\C}{\mathds C} % complex vector space
\providecommand{\R}{} \renewcommand{\R}{\mathds R} % real vector space

\DeclareMathOperator*{\argmin}{arg\,min}		               % argmin
\DeclareMathOperator*{\argmax}{arg\,max}		               % argmax


\newcommand{\mat}[1]{\mathbf{#1}}
\renewcommand{\vec}[1]{\mathbf{#1}}
\newcommand{\norm}[1]{\left\Vert\, #1\, \right\Vert}
\newcommand{\abs}[1]{\left\vert\, #1\, \right\vert}
\newcommand{\paren}[1]{\left(#1\right)} % parenthesis 
\newcommand{\sqbr}[1]{\left[#1\right]} % square bracket
\newcommand{\cubr}[1]{\left\{#1\right\}} % curly bracket
\newcommand{\IProd}[2]{\left\langle\,#1,\,#2\,\right\rangle}


\newcommand{\Rank}{\mathsf{Rank}\;}
\newcommand{\Sign}{\mathsf{sign}\;}
\newcommand{\Ker}{\mathsf{Ker}\;}
\newcommand{\Span}{\mathsf{span}\;}
\newcommand{\Diag}{\mathsf{diag}\;}
\newcommand{\Dim}{\mathsf{dim}\;}
\renewcommand{\Im}{{\mathsf{Im}\,}}
\newcommand{\sphangle}{{\sphericalangle\,}}
\newcommand{\Trace}{\mathsf{Trace}\;}
\renewcommand{\det}{\mathsf{det}\;}
\newcommand{\grad}{\mathsf{grad}\;}
\newcommand{\T}{\mathsf{T}}

